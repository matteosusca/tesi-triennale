\documentclass[12pt,a4paper,openright,twoside]{book}
\usepackage[utf8]{inputenc}
\usepackage{disi-thesis}
\usepackage{code-lstlistings}
\usepackage{notes}
\usepackage{shortcuts}
\usepackage{acronym}

\school{\unibo}
\programme{Corso di Laurea Triennale in Ingegneria e Scienze Informatiche}
\title{Tecnologie di Desktop Remoto e il loro Funzionamento}
\author{Matteo Susca}
\date{\today}
\subject{Reti di Telecomunicazioni}
\supervisor{Prof. Franco Callegati}
\session{I}
\academicyear{2023-2024}

% Definition of acronyms
\acrodef{IoT}{Internet of Thing}
\acrodef{vm}[VM]{Virtual Machine}


\mainlinespacing{1.241} % line spacing in mainmatter, comment to default (1)

\begin{document}

\frontmatter\frontispiece

\begin{abstract}	
Max 2000 characters, strict.
\end{abstract}

\begin{dedication} % this is optional
Optional. Max a few lines.
\end{dedication}

%----------------------------------------------------------------------------------------
\tableofcontents   
% \listoffigures     % (optional) comment if empty
% \lstlistoflistings % (optional) comment if empty
%----------------------------------------------------------------------------------------

% \mainmatter

%----------------------------------------------------------------------------------------
\chapter{Introduzione}
\label{chap:introduction}

Al giorno d'oggi, è sempre più comune incontrare realtà lavorative che offrono la possibilità di lavorare da remoto. La possibilità di connettersi a un computer aziendale da qualsiasi dispositivo connesso a internet comporta numerosi e innegabili vantaggi, come l'accesso diretto alla rete aziendale, l'utilizzo di hardware specifico per alcune attività senza doverlo trasportare fisicamente, e la flessibilità di lavorare da casa o in mobilità. Tutto questo è reso possibile dalle tecnologie di desktop remoto, che permettono di controllare un computer da un altro dispositivo, ovunque esso si trovi, come se ci si trovasse fisicamente davanti ad esso.

Questo lavoro di tesi si propone di analizzare le tecnologie di desktop remoto, partendo dalle loro origini e arrivando alle applicazioni attuali e future. Alla base del desktop remoto e di tecnologie simili vi è il concetto di virtualizzazione del desktop. La virtualizzazione del desktop consiste nella separazione dell'ambiente desktop da un dispositivo fisico attraverso un modello client-server. In questo modello, il desktop virtualizzato viene memorizzato su un server remoto centrale e non sul dispositivo dell'utente. L'utente può quindi accedere a file, applicazioni e dati da qualsiasi dispositivo compatibile, come un PC, un tablet o uno smartphone. % definizione da Desktop Virtualization Frederic P. Miller
Quando invece di un server centrale si utilizza un altro computer come host, si parla specificamente di desktop remoto. Questa tecnologia consente di gestire e controllare da remoto un computer come se ci si trovasse di fronte ad esso fisicamente.

Le tecnologie di desktop remoto hanno subito notevoli cambiamenti nel corso degli anni, evolvendosi in base alle esigenze emergenti e ai progressi tecnologici. Oggi esistono numerosi protocolli tra cui scegliere, ciascuno con le proprie caratteristiche e peculiarità, rispondendo alle diverse esigenze di utilizzo.


%----------------------------------------------------------------------------------------

\chapter{Evoluzione delle Tecnologie di Desktop Remoto}
\label{chap:evolution}

\section{Origini prime Tecnologie}

\section{Analisi e Funzionamento delle Principali Tecnologie di Desktop Remoto}

\subsection{X Window System}

\subsection{Remote Desktop Protocol}

\subsection{Virtual Network Computing}

\subsection{...}

\chapter{Sviluppi e Applicazioni attuali delle Tecnologie di Desktop Remoto}

\section{Nuove Esigenze e Utilizzi}

\section{Protocolli di Desktop Remoto moderni}

\section{Cloud Gaming e Low-latency Streaming}

\section{...}

\chapter{Prospettive, Sviluppi e Applicazioni futuri del Desktop Remoto}

\section{Desktop as a Service}

\section{Edge Computing}

\section{...}

\chapter{Conclusioni}
\label{chap:conclusions}

%----------------------------------------------------------------------------------------
% BIBLIOGRAPHY
%----------------------------------------------------------------------------------------

\backmatter

\nocite{*} % Remove this as soon as you have the first citation

\bibliographystyle{alpha}
\bibliography{bibliography}

\begin{acknowledgements} % this is optional
Optional. Max 1 page.
\end{acknowledgements}

\end{document}
