\documentclass[12pt,a4paper,openright,twoside]{book}
\usepackage[utf8]{inputenc}
\usepackage{disi-thesis}
\usepackage{code-lstlistings}
\usepackage{notes}
\usepackage{shortcuts}
\usepackage{acronym}

\school{\unibo}
\programme{Corso di Laurea Triennale in Ingegneria e Scienze Informatiche}
\title{Tecnologie di Desktop Remoto e il loro Funzionamento}
\author{Matteo Susca}
\date{\today}
\subject{Reti di Telecomunicazioni}
\supervisor{Prof. Franco Callegati}
\session{I}
\academicyear{2023-2024}

% Definition of acronyms
\acrodef{IoT}{Internet of Thing}
\acrodef{vm}[VM]{Virtual Machine}


\mainlinespacing{1.241} % line spacing in mainmatter, comment to default (1)

\begin{document}

\frontmatter\frontispiece

\begin{abstract}	
Max 2000 characters, strict.
\end{abstract}

\begin{dedication} % this is optional
Optional. Max a few lines.
\end{dedication}

%----------------------------------------------------------------------------------------
\tableofcontents   
% \listoffigures     % (optional) comment if empty
% \lstlistoflistings % (optional) comment if empty
%----------------------------------------------------------------------------------------

\mainmatter

%----------------------------------------------------------------------------------------
\chapter{Introduzione}
\label{chap:introduction}

Al giorno d'oggi, è sempre più comune incontrare realtà lavorative che offrono la possibilità di lavorare da remoto. La possibilità di connettersi a un computer aziendale da qualsiasi dispositivo connesso a internet comporta numerosi e innegabili vantaggi, come l'accesso diretto alla rete aziendale, l'utilizzo di hardware specifico per alcune attività senza doverlo trasportare fisicamente, e la flessibilità di lavorare da casa o in mobilità. Tutto questo è reso possibile dalle tecnologie di desktop remoto, che permettono di controllare un computer da un altro dispositivo, ovunque esso si trovi, come se ci si trovasse fisicamente davanti ad esso.

Anche nell'industria videoludica le tecnologie di desktop remoto ricoprono un ruolo importante. Il cloud gaming, ad esempio, è un servizio che permette di giocare a videogiochi in streaming, sfruttando la potenza di calcolo di un server remoto. Il colosso tecnologico NVIDIA, uno dei principali attori in questo settore, ha dimostrato fin dal 2013 con il servizio NVIDIA GRID di essere in grado di offrire un'esperienza di gioco fluida e di alta qualità, grazie all'utilizzo di tecnologie di desktop remoto. Ora il servizio è stato ribattezzato GeForce NOW e offre la possibilità di giocare a numerosi titoli di successo, anche su dispositivi mobili. Anche altre aziende come Google, Microsoft e Amazon stanno investendo in questo settore, con servizi come Google Stadia (attualmente chiuso), Xbox Cloud Gaming e Amazon Luna.

L'assistenza remota è un altro campo in cui le tecnologie di desktop remoto hanno avuto un impatto positivo. I tecnici possono ora accedere al computer di un cliente e risolvere problemi senza doversi recare fisicamente sul posto, riducendo i tempi e i costi di intervento, risultando quindi un vantaggio per entrambe le parti.

Questo lavoro di tesi si propone di analizzare le tecnologie di desktop remoto, partendo dalle loro origini e arrivando alle applicazioni attuali e future. Alla base del desktop remoto e di tecnologie simili vi è il concetto di virtualizzazione del desktop. La virtualizzazione del desktop consiste nella separazione dell'ambiente desktop da un dispositivo fisico attraverso un modello client-server. In questo modello, il desktop virtualizzato viene memorizzato su un server remoto centrale e non sul dispositivo dell'utente. L'utente può quindi accedere a file, applicazioni e dati da qualsiasi dispositivo compatibile, come un PC, un tablet o uno smartphone. % definizione da Desktop Virtualization Frederic P. Miller
Quando invece di un server centrale si utilizza un altro computer come host, si parla specificamente di desktop remoto. Questa tecnologia consente di gestire e controllare da remoto un computer come se ci si trovasse di fronte ad esso fisicamente.

Le tecnologie di desktop remoto hanno subito notevoli cambiamenti nel corso degli anni, evolvendosi in base alle esigenze emergenti e ai progressi tecnologici. Oggi esistono numerosi protocolli tra cui scegliere, ciascuno con le proprie caratteristiche e peculiarità, rispondendo alle diverse esigenze di utilizzo.


%----------------------------------------------------------------------------------------

\chapter{Evoluzione delle Tecnologie di Desktop Remoto}
\label{chap:evolution}
Il voler connettersi da remoto ad un pc è una necessità che si è manifestata fin dai primi anni dell'informatica. Già negli anni '70, insieme al progetto ARPANET, si iniziò a pensare a come poter accedere a terminali remoti tramite una connessione di rete, successivamente implementata con il protocollo Telnet. 
Ovviamente ai giorni nostri, in molti casi, non basta più accedere solamente alla shell di un computer remoto, ma è necessario poter interagire con un'interfaccia grafica. Infatti le esigenze di accesso remoto sono cambiate col tempo e continuano a cambiare. Sono inoltre diverse a seconda del contesto in cui ci si trova: un utente che lavora da casa ha esigenze diverse da un gamer che vuole giocare in mobilità, o da un tecnico che deve risolvere un problema su un computer remoto. Per questo motivo, e per la continua evoluzione delle tecnologie, nel corso degli anni sono stati sviluppati molteplici protocolli e tecnologie con funzionalità e caratteristiche diverse.

\section{Origini prime Tecnologie}
Negli anni '80 esistavo già tecnologie che permettevano di accedere ad un terminale remoto da una macchina locale. Esistevano macchine adibite solamente alla connessione con un terminale remoto; queste macchine erano chiamate "dumb terminal" (terminale stupido) o "thin client". Quest'ultimo termine è ancora utilizzato oggi per indicare un dispositivo che si connette ad un server remoto per eseguire applicazioni e accedere a risorse di rete.
Con l'introduzione e aumento di sistemi operativi con interfaccia grafica, era inevitabile la conseguente evoluzione delle tecnologie di accesso remoto. Fu infatti nel 1984 che fu introdotto, dal Massachusetts Institute of Technology, il X Window System, successore del W Window System.


\section{Analisi e Funzionamento delle Principali Tecnologie di Desktop Remoto}
In questo capitolo analizzeremo alcune delle principali tecnologie di desktop remoto presentando il contesto storico in cui sono state sviluppate, le caratteristiche principali e il funzionamento.

\subsection{X Window System}
L'X Window System, noto anche come X o X11, è un sistema di finestre che consente di eseguire applicazioni grafiche localmente o su computer remoti tramite una archittettura client-server.
Nasce dall'esigenza comune di due progetti del MIT, Athena e il Laboratory for Computer Science, di avere un sistema che permettesse di accedere a risorse grafiche distribuite su una rete di computer eterogenei, indipendentemente dall'hardware o dal sistema operativo utilizzato.
Il nome "X" deriva dal suo predecessore, il W Window System, sviluppato presso l'Università di Stanford. A differenza del W Window System, X introduceva la possibilità di gestire applicazioni grafiche su workstation remote, favorendo il concetto di separazione client-server per l'accesso ai display.

La versione attuale di X è la X11, rilasciata nel 1987. Questa versione è ancora ampiamente utilizzata e supportata, nonostante siano state sviluppate versioni successive come X.Org e XFree86.

\subsubsection{Architettura del sistema}
% fonte: https://www.maketecheasier.com/the-x-window-system/ e wikipedia

X è una collezione di software che si posizionano tra l'hardware (o per meglio dire il kernel) e altri software di più alto livello detti X-clients. Il sistema X si basa su una archittettura client-server dove il server si trova sul compute che ospita l'interfaccia grafica e il client è l'applicazione che richiede servizi grafici. Questa terminologia può sembrare controintuitiva in quanto chi non ha familiarità con il sistema X potrebbe pensare che sia invertita. In realtà X si mette nei panni delle applicazioni che richiedono servizi grafici e I/O al server X (che appunto interagisce con l'hardware).
I client possono essere locali o remoti. Ogni server X può connettersi a molteplici client 
% immagine architettura X Window System

\subsubsection{Funzionamento generale}

Lorem ipsum dolor sit amet, consectetur adipiscing elit. Nullam nec purus nec nunc ultricies tincidunt. Nullam nec purus nec nunc ultricies tincidunt. Nullam nec purus nec nunc

\subsubsection{Desktop remoto in X Window System}

La comunicazione tra server a client avviene in modalità full-duplex, ovvero entrambi possono inviare e ricevere dati contemporaneamente. Lo scambio di messaggi avviene """canonicamente""" attraverso il protocollo TCP/IP, ma spesso vengono utilizzati altri canali comunicativi come Unix domain sockets o memoria condivisa. Indipendentemente dal canale utilizzato è necessario che questo sia affidabile e che garantisca l'integrità e l'ordine dei messaggi, il protocollo X non implementa nativamente meccanismi di ritrasmissione o riordino dei pacchetti.

\begin{thebibliography}{9}
\bibitem{scheifler1986x} 
Robert W. Scheifler and Jim Gettys, 
\textit{The X Window System}, 
MIT Laboratory for Computer Science, Digital Equipment Corporation, 1986.
\end{thebibliography}


\subsection{Remote Desktop Protocol}

\subsection{Virtual Network Computing}

\subsection{...}

\chapter{Sviluppi e Applicazioni attuali delle Tecnologie di Desktop Remoto}

\section{Nuove Esigenze e Utilizzi}

\section{Protocolli di Desktop Remoto moderni}

\section{Cloud Gaming e Low-latency Streaming}

\section{...}

\chapter{Prospettive, Sviluppi e Applicazioni futuri del Desktop Remoto}

\section{Desktop as a Service}

\section{Edge Computing}

\section{...}

\chapter{Conclusioni}
\label{chap:conclusions}

%----------------------------------------------------------------------------------------
% BIBLIOGRAPHY
%----------------------------------------------------------------------------------------

\backmatter

\nocite{*} % Remove this as soon as you have the first citation

\bibliographystyle{alpha}
\bibliography{bibliography}

\begin{acknowledgements} % this is optional
Optional. Max 1 page.
\end{acknowledgements}

\end{document}
