\documentclass[12pt,a4paper,openright,twoside]{book}
\usepackage[utf8]{inputenc}
\usepackage{disi-thesis}
\usepackage{code-lstlistings}
\usepackage{notes}
\usepackage{shortcuts}
\usepackage{acronym}

\school{\unibo}
\programme{Corso di Laurea Triennale in Ingegneria e Scienze Informatiche}
\title{Tecnologie di Desktop Remoto e il loro Funzionamento}
\author{Matteo Susca}
\date{\today}
\subject{Reti di Telecomunicazioni}
\supervisor{Prof. Franco Callegati}
\session{I}
\academicyear{2023-2024}

% Definition of acronyms
\acrodef{IoT}{Internet of Thing}
\acrodef{vm}[VM]{Virtual Machine}


\mainlinespacing{1.241} % line spacing in mainmatter, comment to default (1)

\begin{document}

\frontmatter\frontispiece

\begin{abstract}	
Max 2000 characters, strict.
\end{abstract}

\begin{dedication} % this is optional
Optional. Max a few lines.
\end{dedication}

%----------------------------------------------------------------------------------------
\tableofcontents   
\listoffigures     % (optional) comment if empty
\lstlistoflistings % (optional) comment if empty
%----------------------------------------------------------------------------------------

\mainmatter

%----------------------------------------------------------------------------------------
\chapter{Introduzione}
\label{chap:introduction}
%----------------------------------------------------------------------------------------

\chapter{Evoluzione delle Tecnologie di Desktop Remoto}
\label{chap:evolution}

\section{Origini prime Tecnologie}

\section{Analisi e Funzionamento delle Principali Tecnologie di Desktop Remoto}

\subsection{X Window System}

\subsection{Remote Desktop Protocol}

\subsection{Virtual Network Computing}

\subsection{...}

\chapter{Sviluppi e Applicazioni attuali delle Tecnologie di Desktop Remoto}

\section{Nuove Esigenze e Utilizzi}

\section{Protocolli di Desktop Remoto moderni}

\section{Cloud Gaming e Low-latency Streaming}

\section{...}

\chapter{Prospettive, Sviluppi e Applicazioni futuri del Desktop Remoto}

\section{Desktop as a Service}

\section{Edge Computing}

\section{...}

\chapter{Conclusioni}
\label{chap:conclusions}

%----------------------------------------------------------------------------------------
% BIBLIOGRAPHY
%----------------------------------------------------------------------------------------

\backmatter

\nocite{*} % Remove this as soon as you have the first citation

\bibliographystyle{alpha}
\bibliography{bibliography}

\begin{acknowledgements} % this is optional
Optional. Max 1 page.
\end{acknowledgements}

\end{document}
